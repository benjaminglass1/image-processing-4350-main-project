\documentclass{article}


%%% Preamble %%%
%
%
%

\usepackage{babel}			% 	Expands text mode
\usepackage{csquotes}			% 	Permits \enquote{quote} for quick quote encapsulation
\usepackage{graphicx}			% 	Permits \includegraphics[size = x]{image}
\usepackage{caption}			% 	Permits the void caption \caption*{}
\usepackage{a4wide}			% 	Expand width of body lazily

			%	Makes Table of Contents Functionally a table of Hyperlinks
\usepackage{hyperref}
\hypersetup{
	colorlinks,
	citecolor= black,
	filecolor = black,
	linkcolor = blue,
	urlcolor = blue
	}

			%	Next Three Lines permit use of .tif images in \includegraphics
\usepackage{epstopdf}
\epstopdfDeclareGraphicsRule{.tif}{png}{.png}{convert #1 \OutputFile}
\AppendGraphicsExtensions{.tif}


\title{Final Project \\ EENG 4350.034 \\ Image Processing \\ Dr. Ononye}
\author{Gibbons, Glass, \& Rodriguez}
\date{04/27/2019}


%
%
%
%%% End of Preamble %%%


%%% Body Document %%%
%
%
%

\begin{document}

\maketitle

\pagenumbering{gobble}

\hfill \\
\hfill \\
\hfill \\
\hfill \\
\hfill \\
\hfill \\
\hfill \\
\hfill \\
\hfill \\
\hfill \\
\hfill \\
\hfill \\
\hfill \\
\hfill \\
\hfill \\
\hfill \\
\hfill \\
\hfill \\
\hfill \\
\hfill \\
\hfill \\
\hfill \\
\hfill \\
\hfill \\
\hfill \\
\hfill \\
\hfill \\
\hfill \\
\hfill \\
\hfill \\

%\abstract{}
\begin{abstract}

%	The Abstract	should contain:
%	First and foremostly: The information required to quickly inform somebody :
%					whether or not 
%			Reading your paper is a waste of their time.
%
%	Any related information that would benefit someone who reads your document
%	Any topics covered within the document and to what level the contents are written
%		such as for the benefit of:
%		students, researchers, professors, engineers, professionals
%
%		MOST IMPORTANTLY OF ALL: 
%			R E S P E C T
%		  the time of your reader
%
%	An abstract should take less than 2-5 minutes to read depending on the professional field
%	Students should never have an abstract that takes longer than 2 minutes to read.
%		NEVER
%


\end{abstract}


\pagebreak

\tableofcontents

\pagenumbering{roman}

\pagebreak

\section{Introduction}

% 	The Introduction should contain:
% 	the group's mission, from who the project was given and under what circumstance
% 	The circumstances include but may not be limited to:
% 	Course Objective {per stated in syllabus or procedure provided - reference to the text properly inside the introduction}
%	The intended audience if not explicit in the syllabus or procedure ** This should already be stated in the abstract
%	


\pagenumbering{arabic}



\pagebreak

\section{Theory}

%	The Theory should contain:
%	Reference to the particular concepts in the field that are immediately drawn from
%
%	For example, an experiment involving testing logic ICs on a breadboard involves
%		The application of Ohm's Law & the Principles of Logical Definitions
%			such as Ohm's Law:
%				1. Resistivity & Conductivity
%				2. Voltage Drop relationship to conductivity
%			such as Principles of Logical Definition
%				1. AND includes both terms completely
%				2. OR includes at least one term
%				3. eXclusive-OR (XOR) includes ONLY one term
%				4. NOT is true only if the formentioned term is not present
%			and maybe including such in binary terms of 0's & 1's.
%



%%%%%%%%%%%%%%%%%%%%%%%%%%%
%									%
%		Personal Outline For Project			%
%									%
%%%%%%%%%%%%%%%%%%%%%%%%%%%
%
%
%	1. 	Gaussian Noise
%	2.	RGB file construction
%	3.	HSI Utilization
%	4. 	Adaptive Noise Reduction *Ch 5 Section 3*
%
%

\noindent
The theory behind this project is four fold:\\
\indent 
1) Inspecting the Gaussian Noise Model .\\
\indent
2) Exegesis on RGB image structure (Red, Green, \& Blue). \\
\indent
3) The application of HSI Color Model Extrapolation (RGB to Hue, Saturation, \& Intensity).\\
\indent
4) The utility of Adaptive Noise Reduction.

\subsection*{Gaussian Noise Model}
\noindent
The Gaussian Noise Model is a practical way of simulating noise in many circumstances across many fields.\\
\hfill \\
It is derived from the Gaussian Statistical Model, shown in the following equation as well as figure:\\
\[ \scalebox{2}{$ $\qquad \qquad \quad$
pdf(x) = \frac{1}{\sigma \sqrt{2 \cdot \pi}} e^{\frac{1}{2}\big(\frac{x-\mu}{\sigma}\big)^{2}} $\qquad \qquad \quad (1)$
$}\]

\begin{figure}[h!]
\includegraphics[scale = 0.38]{miscImages/GaussExample.jpg}
\caption{Gaussian Curve Generated In MATLAB}
\end{figure}

\pagebreak
\noindent
Gaussian Noise may take place via overlapping signals that are themselves transmitted in any mode similar to
oscillating sinusoidal signals, and may generate white noise.  This is form of white noise taught to University 
Physics II students with Gaussian spectrum distribution curves spanning the spectrum of visible light.  Normally 
with Red, Green, and Blue Gaussian spectrum distributions overlapping each other.\\
\hfill \\
Gaussian Noise may further be understood as sensor/signal perturbations that oscillate about the mean or true,
desired value.  This perturbation creates an effect whereby the most likely events are the central points of oscillation
unto the effect that the mean is the most likely actual signal.  Hence, one may achieve this effect by applying a 
Gaussian Noise Filter over the signal.\\
\hfill \\
However, the particular Gaussian Noise applied in this project is that across a two dimensional matrix, namely: a
photograph or image.

\subsection*{RGB Image Structure}

RGB is an acronym for the Red, Green, and Blue Color Model.\\
\hfill \\




\subsection*{HSI}


\subsection*{Adaptive Noise Reduction}

%\pagebreak

%\section{Procedure}

%	The Procedure should contain all necessary actions in order to acheive your result

\pagebreak

\section{Methodology}

%	The Methodology should contain but may not be limited to:
%		The general method that is applied to each portion of the procedure.
%		Any techniques applied to obtain the results achieved
%		Point to discussion if there were attempted techniques that failed to achieve the desired result
%			and were not used to obtain your final results.
%		THEN discuss their implications in the Discussion Section
%
%	The Methodology may also include a ``Calculations Section'' 
%		wherein, one demonstrates the calculation processes required to obtain the desired results.
%		This section may be used to present calculations and referred back to in the results section when 
%		being cited.
%




\pagebreak

\section{Results}

%	The Results includes the final results of any calculations, images, or experiments
%	The Results should have very limited elaboration::
%			Let the Results Speak for Themselves
%		Otherwise,
%			Compile your results so that they are more easily understood
%	Save any deep inspection for the discussion
%


\pagebreak

\section{Discussion}

%	See all the above and use your imagination
%
%

\pagebreak

\section{Conclusion}

%	The conclusion should be a concise summary of the results and discussion.
%
%	The conclusion should assert whether or not you accomplished the goal of the experiment
%	and with what margins or error and uncertainty.
%


\end{document}